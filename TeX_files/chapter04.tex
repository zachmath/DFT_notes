\chapter{Temperature-dependent Hartree-Fock Bogoliubov}

A lot of these ideas I'm getting from \cite{Schunck2015} as well as Nicolas' own temperature-dependent HFB notes.

\section*{A brief overview of the theory}

As in any statistical theory, one first must determine which sort of ensemble properly describes the system. Nuclei have (in principle) conserved number of particles; however in HFB theory, that's somewhat flexible since the BCS transformation explicitly breaks particle number symmetry. In principle we should perhaps use a microcanonical ensemble to describe a nucleus as a closed, isolated system, but that turns out to be challenging to solve because it requires a full knowledge of the eigenspectrum of the nucleus. Using that quirk of HFB theory, we wiggle our way out of this hairiness\footnote{You can wave your hands here and say that finite temperatures let you break superfluid pairs, and so the number of "quasiparticles" (which you could argue might have referred to pairs before but now might also include individual particles) can change.} to instead describe our system using the grand canonical ensemble, and this turns out to be tractable.

Moving forward by minimizing the grand potential $\Omega$ gives us for the density:
\begin{equation*}
\hat{D} = \frac{1}{Z}e^{-\beta\left(\hat{H}-\mu\hat{N}\right)}
\end{equation*}

\noindent with associated partition function

\begin{equation*}
Z = Tr\left[e^{-\beta\left(\hat{H}-\mu\hat{N}\right)}\right]
\end{equation*}

Getting specifically to our particular choice of mean-field Hamiltonian, we substitute in some one-body operator for the exponent:

\begin{equation*}
\hat{D}_{HF} = \frac{1}{Z}e^{-\beta\hat{K}}, Z = Tr\left[e^{-\beta\hat{K}}\right]
\end{equation*}

\noindent where in the plain ol' Hartree Fock case, $\hat{K} = \sum_{ij}K_{ij}c_i^\dagger c_j$ (in the HFB case, $\hat{K}$ is a sum of all different one-body operator types, but it's the same basic idea).

Defining the HF density matrix $\rho_{ij}=Tr\left[\hat{D}_{HF}c_j^\dagger c_i\right]$, we can show the following useful correspondence relations:

\begin{align*}
\rho &= \frac{1}{1+e^{\beta\hat{K}}} \\
Tr\left[\hat{D}_{HF}\hat{A}\right] &= tr\left[\rho\hat{A}\right] = \sum_{ij}\rho_{ij}\hat{A}_{ij}
\end{align*}

\noindent where $\hat{A}$ is some operator in the single-particle basis. Similar things happen for the HFB case. At the end of the day in HFB, things work out to be pretty similar to the way they were before, except the density in the quasiparticle basis is replaced by

\begin{equation*}
\mathcal{R} =
\left(\begin{array}{cc}
0 & 0 \\
0 & 1
\end{array}\right)
\rightarrow
\left(\begin{array}{cc}
f & 0 \\
0 & 1-f
\end{array}\right)
\end{equation*}

\noindent Obviously there's a lot more richness to it than that, but this helps to at least see the basic skeleton of what changes at finite temperature.

\section*{Temperature-Dependent ATDHFB}

Let us quickly review the essence of Time-Dependent Hartree-Fock-Bogoliubov (TDHFB). The fundamental assumption of TDHFB is that a system which is a Slater determinant at time $t=0$ and which is then allowed to evolve in time will remain a Slater determinant at all times $t$. This assumption allows us to write to TDHFB equation:

\begin{equation*}
i\hbar \mathcal{\dot{R}} = \left[\mathcal{H},\mathcal{R}\right]
\end{equation*}

\noindent where in the single-particle basis

\begin{equation*}
\mathcal{H} = 
\left(\begin{array}{cc}
h-\lambda & \Delta \\
-\Delta^* & -h^*+\lambda
\end{array}\right), 
\qquad \mathcal{R} = 
\left(\begin{array}{cc}
\rho & \kappa \\
-\kappa^* & 1-\rho^*
\end{array}\right)
\end{equation*}

The \textit{additional} assumption that collective motion is slow compared to single particle motion of the system is called the \textit{adiabtic approximation}, and the consequent model is called Adiabatic Time-Dependent Hartree-Fock-Bogoliubov (ATDHFB). Historically, the reason for this assumption comes from microscopic-macroscopic models of nuclear fission, where the dynamics of the system are described by a few collective shape variables and their derivatives (you might think of them semiclassically as coordinates and velocities). The adiabatic approximation is implicit in this assumption. ATDHFB provides the bridge for bringing this useful framework into a self-consistent, fully-microscopic picture.

Once the system is described in terms of collective coordinates and velocities, the energy can be expressed as the sum of a "potential" term (which depends on the coordinates) and a "kinetic" term (which depends on the velocities). Our goal is to understand the kinetic part of the energy, which in some sense describes the dynamics of, for example, a fissioning nucleus, in terms of the first few multipole moments of the nucleus. A key component of this will be the inertia tensor $\mathcal{M}$, which plays the role of the "mass": $E_{kin}~\frac{1}{2}\mathcal{M}\dot{q}^2$

\subsection*{Review of ATDHFB}

With the adiabatic assumption in place, we can write the density as an expansion around some time-even zeroth-order density:

\begin{align*}
\mathcal{R}(t) &= e^{i\chi(t)}\mathcal{R}_0(t)e^{-i\chi(t)} \\
&= \mathcal{R}_0 + \mathcal{R}_1 + \mathcal{R}_2 + \dots
\end{align*}

\noindent where $\chi$ is assumed to be "small" (which is explained more rigorously in \cite{Baranger1978}) and

\begin{align}\label{densities}
\mathcal{R}_1 &= i\left[\chi, \mathcal{R}_0\right] \\
\mathcal{R}_2 &= \frac{1}{2}\left[\left[\chi, \mathcal{R}_0\right], \chi\right] 
\end{align}

\noindent The HFB matrix, being a function of $\mathcal{R}$, is likewise expanded:

\begin{equation*}
\mathcal{H} = \mathcal{H}_0 + \mathcal{H}_1 + \mathcal{H}_2 + \dots
\end{equation*}

\noindent and together $\mathcal{R}$ and $\mathcal{H}$ are plugged into the TDHFB equation. Gathering terms in powers of $\chi$:

\begin{align}\label{ATDHFB}
i\hbar\mathcal{\dot{R}}_0 &= \left[\mathcal{H}_0, \mathcal{R}_1\right] + \left[\mathcal{H}_1, \mathcal{R}_0\right] \\
i\hbar\mathcal{\dot{R}}_1 &= \left[\mathcal{H}_0, \mathcal{R}_0\right] + \left[\mathcal{H}_0, \mathcal{R}_2\right]
 + \left[\mathcal{H}_1, \mathcal{R}_1\right] + \left[\mathcal{H}_2, \mathcal{R}_0\right]
\end{align}

\noindent These two equations are the ATDHFB equations. They can be solved self-consistently to find both $\chi$ and $\mathcal{R}_0$; however, this is rarely done in practice. More commonly what is done is to exploit the fact that solutions to the ATDHFB equations are (by design) \textit{close} to true HFB solutions. We then take HFB solutions and compute their time derivatives by the first ATDHFB equation to get ATDHFB-like behavior without going through the full trouble of ATDHFB.

One nice feature of using true HFB solutions instead of ATDHFB solutions is that the matrix $\mathcal{H}_0$ is diagonal in the HFB basis.

Finally, the total energy of the system is found to be

\begin{equation*}
E(\mathcal{R}) = E_{HFB} + \frac{1}{2}\mathnormal{Tr}\left(\mathcal{H}_0\mathcal{R}_1\right) + \frac{1}{2}\mathnormal{Tr}\left(\mathcal{H}_0\mathcal{R}_2\right) + \frac{1}{4}\mathnormal{Tr}\left(\mathcal{H}_1\mathcal{R}_1\right)
\end{equation*}

\subsection*{Relation between $\dot{\chi}$ and $\dot{\mathcal{R}}$}

Eventually we'll want to express the energy in terms of the multipole moments $q$ and their derivatives, but for now we will content ourselves with expressing the energy in terms of $\mathcal{R}$ and $\dot{\mathcal{R}}$. From the first ATDHFB equation:

\begin{equation*}
i\hbar\mathcal{\dot{R}}_0 = \left[\mathcal{H}_0, \mathcal{R}_1\right] + \left[\mathcal{H}_1, \mathcal{R}_0\right]
\end{equation*}

Working in the HFB quasiparticle basis, we have (at finite temperatures)

\begin{equation*}
\mathcal{H} = 
\left(\begin{array}{cc}
E & 0 \\
0 & -E
\end{array}\right), 
\qquad \mathcal{R} = 
\left(\begin{array}{cc}
f & 0 \\
0 & 1-f
\end{array}\right)
\end{equation*}

\noindent Note that the block matrices $E$ and $f$ are both diagonal. Ultimately, by using the equations \ref{densities}, we arrive at the result:

\begin{align*}
\hbar \dot{\mathcal{R}}_{ab}^{11} &= (E_a-E_b)(f_b-f_a)\chi_{ab}^{11} + (f_b-f_a)\mathcal{H}^{11}_{(1),ab} \\
\hbar \dot{\mathcal{R}}_{ab}^{12} &= (E_a+E_b)\left(1-(f_a+f_b)\right)\chi_{ab}^{12} + \left(1-(f_a+f_b)\right)\mathcal{H}^{12}_{(1),ab} \\
\hbar \dot{\mathcal{R}}_{ab}^{21} &= (E_a+E_b)\left(1-(f_a+f_b)\right)\chi_{ab}^{21} - \left(1-(f_a+f_b)\right)\mathcal{H}^{21}_{(1),ab} \\
\hbar \dot{\mathcal{R}}_{ab}^{22} &= (E_a-E_b)(f_b-f_a)\chi_{ab}^{22} - (f_b-f_a)\mathcal{H}^{22}_{(1),ab}
\end{align*}

It is common (the so-called "cranking approximation") to assume that changes in the density have approximately no effect on the mean field, in which case these relations reduce to

\begin{align*}
\hbar \dot{\mathcal{R}}_{ab}^{11} &= (E_a-E_b)(f_b-f_a)\chi_{ab}^{11} \\
\hbar \dot{\mathcal{R}}_{ab}^{12} &= (E_a+E_b)\left(1-(f_a+f_b)\right)\chi_{ab}^{12} \\
\hbar \dot{\mathcal{R}}_{ab}^{21} &= (E_a+E_b)\left(1-(f_a+f_b)\right)\chi_{ab}^{21} \\
\hbar \dot{\mathcal{R}}_{ab}^{22} &= (E_a-E_b)(f_b-f_a)\chi_{ab}^{22}
\end{align*}

\subsection*{Kinetic Energy at Finite Temperature}

\section*{Fission at finite temperature}

There are several complications associated with considering a nucleus at finite temperature. I'd like to discuss first of all what that even means and why it is significant, and then I'll talk about some of the challenges of this approach.

The idea of considering fission as a finite temperature process stems from trying to develop a picture of induced fission. In induced fission, a neutron which carries some amount of energy is captured by a heavy nucleus in its ground state. That extra energy's gotta go somewhere, but where? In a large nucleus, you have all sorts of places, including any number of single particle excitations, or combination of single particle excitations, or perhaps the entire nucleus moves together as one large collective excitation. Apparently some people went through and did the combinatorics of these possible excitations and decided that the number of them was huge (like, $~10^{12}$ huge) \cite{Hilaire2012}. So handling them explicitly just isn't going to work.

Additionally, DFT might not be the best tool for performing finite-temperature calculations. This is because DFT is typically implemented as a variational method, which means it's good for ground state calculations. But highly-deformed nuclei are inherently \textit{not} in their ground state. In fact, this is true even for "zero-temperature" DFT as well. In practice the results we get are pretty good (most likely the system has time to equilibriate to its "deformed ground state" at each deformation step, or in other words the path to scission proceeds ~adiabatically), but it's something to definitely keep in mind that (I suspect) could strongly affect your half-life predictions and fragment energies in particular. Furthermore, it's not clear to me how you might correct that should the need arise.

A third thing that Nicolas claims is that the temperature should depend on the actual deformation. I'm not sure where exactly this comes from but it sounds plausible to me. The thing that really gets me is that he then does some voodoo black magic to conclude that in FT-DFT, we have to assume that T is constant across the entire PES. That I'm not understanding and I'll have to ask him about that when he gets back.