\chapter{Nuclear Clusters}

Today I'm trying to read and learn what I can about so-called ``clusters.'' It doesn't sound to me like an inherently difficult concept, but apparently there are some things related to cluster formation that are poorly-understood, and besides that I want to see if I can better improve my study habits by basically keeping a ``study journal.'' So here goes...

My introduction to clustering came from a line in a paper by Nicolas \cite{Schunck2014PES}, where he refers to a small channel in the potential energy surface for $^{240}$Pu corresponding to something called ``cluster radioactivity.''

It comes up quite a bit more with a lot more references in Chunli's first localization paper \cite{Zhang2016localization}. The rest of Chunli's paper doesn't go into much more detail about what clusterization \textit{is}, but it describes a technique for observing shell structure in clusters (or anything, really), even within a ``smeared-out'' DFT framework. It's cool because DFT tends to blur and spread out the impact of single-particle wavefunctions across the entire nuclear volume, but this localization technique gives you a way to see something shell-like within the nucleus. That's especially useful in fission because the formation of fragments is driven by the shell structure of the pre-fragments (and not the final fragments themselves). So if you start to see shell structure appearing in your scissioning nucleus, you might be able to say something about what the final fragments are, just based on the shell structure of the pre-fragments.

But I'm still skeptical. $^{86}$Kr and $^{84}$Se will probably have almost identical shell structure (magic $N=50$ for neutrons, and an off-shell even number of protons). Is it really likely that you'd be able to identify the different isotopes just using the localization measure? I actually tested something very much like this back for platinum-176. There I actually used an even more extreme example of $^{84}$Se, $^{82}$Se, and $^{86}$Zr. The seleniums were almost identical (the neutron spatial localization was only very slightly different), and the zirconium was even \textit{pretty} similar (whatever that means). You can see the results somewhere on my computer where I have them stored, and I also showed a few sample figures in the \texttt{research\_notes.tex} file, under \textbf{Nucleon Localization Function \rightarrow 28 September 2017}

But I digress...

Apparently historically, the idea of cluster emission was originally characterized by situations where the emitted fragment was larger than an alpha particle, but capped off at about $Z_e^{max}=28$ (see the introduction to \cite{Warda2011} for kind of a historical overview of cluster radiation). However, starting in around 2012 and taking into account results from heavy and superheavy systems, the criteria was relaxed to allow for $Z_e>28$. They had noticed that very often the larger remaining fragment was $^{208}$Pb or one of its neighbors, and so now cluster emission includes heavy fragments up to $Z_e^{max}=Z-82$ (see \cite{Poenaru2012}). Physically, it's basically the same thing happening (lead sheds whatever is left and the rest forms a cluster); the only major difference is that superheavy isotopes can produce heavier clusters. Several theories have been proposed to describe the phenomenon, such as superasymmetric fission or tunneling of a preformed fragment through a potential barrier. We actually have a bit of useful information we can use to test that. If you look in my \texttt{research\_notes.tex} file, under \textbf{Nucleon Localization Function \rightarrow 29 September 2017} you can see some nice diagrams that show the development of shell structure in the fragments of $^{294}$Og. Based solely on this set of results, I don't feel comfortable with the idea of a preformed fragment tunneling through a potential barrier - at least not for the smaller fragment. Look at the proton localization for krypton and you'll see that its shell structure doesn't develop until fairly late. The lead develops fairly quickly, however, and I believe that was the conclusion reached in \cite{Warda2011}.