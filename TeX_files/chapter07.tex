\chapter{Nuclear Clusters}

Today I'm trying to read and learn what I can about so-called ``clusters.'' It doesn't sound to me like an inherently difficult concept, but apparently there are some things related to cluster formation that are poorly-understood, and besides that I want to see if I can better improve my study habits by basically keeping a ``study journal.'' So here goes...

My introduction to clustering came from a line in a paper by Nicolas \cite{Schunck2014PES}, where he refers to a small channel in the potential energy surface for $^{240}$Pu corresponding to something called ``cluster radioactivity.''

It comes up quite a bit more with a lot more references in Chunli's first localization paper \cite{Zhang2016localization}. The rest of Chunli's paper doesn't go into much more detail about what clusterization \textit{is}, but it describes a technique for observing shell structure in clusters (or anything, really), even within a ``smeared-out'' DFT framework. It's cool because DFT tends to blur and spread out the impact of single-particle wavefunctions across the entire nuclear volume, but this localization technique gives you a way to see something shell-like within the nucleus. That's especially useful in fission because the formation of fragments is driven by the shell structure of the pre-fragments (and not the final fragments themselves). So if you start to see shell structure appearing in your scissioning nucleus, you might be able to say something about what the final fragments are, just based on the shell structure of the pre-fragments.

But I'm still skeptical. $^{86}$Kr and $^{84}$Se will probably have almost identical shell structure (magic $N=50$ for neutrons, and an off-shell even number of protons). Is it really likely that you'd be able to identify the different isotopes just using the localization measure? I actually tested something very much like this back for platinum-176. There I actually used an even more extreme example of $^{84}$Se, $^{82}$Se, and $^{86}$Zr. The seleniums were almost identical (the neutron spatial localization was only very slightly different), and the zirconium was even \textit{pretty} similar (whatever that means). You can see the results somewhere on my computer where I have them stored, and I also showed a few sample figures in the \texttt{research\_notes.tex} file, under \textbf{Nucleon Localization Function $\rightarrow$ 28 September 2017}

But I digress...

Apparently historically, the idea of cluster emission was originally characterized by situations where the emitted fragment was larger than an alpha particle, but capped off at about $Z_e^{max}=28$ (see the introduction to \cite{Warda2011} for kind of a historical overview of cluster radiation). However, starting in around 2012 and taking into account results from heavy and superheavy systems, the criteria was relaxed to allow for $Z_e>28$. They had noticed that very often the larger remaining fragment was $^{208}$Pb or one of its neighbors, and so now cluster emission includes heavy fragments up to $Z_e^{max}=Z-82$ (see \cite{Poenaru2012}). Physically, it's basically the same thing happening (lead sheds whatever is left and the rest forms a cluster); the only major difference is that superheavy isotopes can produce heavier clusters. Several models have been proposed to describe the phenomenon, such as superasymmetric fission or tunneling of a preformed fragment through a potential barrier. We actually have a bit of useful information we can use to test that. If you look in my \texttt{research\_notes.tex} file, under \textbf{Nucleon Localization Function $\rightarrow$ 29 September 2017} you can see some nice diagrams that show the development of shell structure in the fragments of $^{294}$Og. Based solely on this set of results, I don't feel comfortable with the idea of a preformed fragment tunneling through a potential barrier - at least not for the smaller fragment. Look at the proton localization for krypton and you'll see that its shell structure doesn't develop until fairly late. The lead develops fairly quickly, however, and I believe that was the conclusion reached in \cite{Warda2011}.

Michal Warda gave a[n unpublished] presentation in September 2017 about cluster radioactivity in superheavy nuclei. He showed some cool pictures of potential energy surfaces evolving from actinides to superheavies, and you can totally see cluster emission go from a very unlikely branch to perhaps the dominant fission channel. Ultimately his conclusions are:

\begin{itemize}
\item Asymmetric fission in superheavy nuclei region has the same nature as cluster radioactivity in actinides
\item This decay may be dominant in some superheavy nuclei
\item Sharp fragment mass distribution with $^{208}$Pb fragment is predicted
\end{itemize}

Cluster formation in actinides generally seems to be associated with a fairly long half-life (typically $10^{11}-10^{26}$ seconds, according to Warda's presentation; see also \cite{Poenaru2011} and references therein) and a low branching ratio relative to $\alpha$ decay ($10^{-9}-10^{-16}$). However, he also shows a drastic reduction in the size of the barrier to cluster formation, from upwards of 25 MeV down to around 5 MeV, and I suspect the half-lives will see a major reduction as a consequence. It's impossible to say for sure without actually \textit{doing} the calculation, but I suspect just from looking at the barrier that it'll be comparable with asymmetric fission in actinides or shorter. Figure 4 in \cite{Poenaru2011} makes it look like $\alpha$ decay will win out over cluster radioacitivity only \textit{just barely}, with a half-life only perhaps a factor of 10 shorter.

Also worth noting is that ``Several attempts to detect $^{12}$C radioactivity of the neutron deficient $^{114}$Ba
with a daughter in the neighborhood of the double magic $^{100}$Sn, predicted to have a larger $b_\alpha$, have failed'' \cite{Poenaru2011}. So it seems to be a phenomenon linked to lead or, more likely, heavy and superheavy elements. Seeing that for superheavy elements, cluster emission and asymmetric fission appear to be one and the same, it makes sense to think of cluster emission as a particularly asymmetric instance of regular, garden-variety fission.

\subsection*{A note about ``clusters''}
The notion of a ``cluster'' is not, so far as I can tell, rigorously-defined. In all I've said leading up to this, cluster radiation/cluster emission is the process by which lead (or some lead-like nucleus) sheds its excess nucleons, and whatever is left just forms into its own so-called ``cluster.'' So in that case, a cluster is just a clump of leftover nucleons.

But as I say, that is not totally rigorous. Bastian and Witek published a paper (I should say, they \textit{submitted} a paper, because at the time I am writing this, it is only available on the arXiv: https://arxiv.org/abs/1710.00579) where they discussed ``cluster formation in pre-compound nuclei.'' Their argument was that entrance channel effects can't necessarily be ignored in various collisions involving compound nuclei, because sometimes the nucleus isn't as thermalized as we'd like to think (we haven't lost all the information about the incoming particles). $\alpha$ particles in particular like to stick together and move around in clumps. So in that case, ``clusters'' refers to substructures within the overall system (like $\alpha$ particles hanging out at the tips of an elongated nucleus, or rings of carbon-12 forming near the middle).

There's probably no real need to overthink this: ``cluster'' is just a useful word that describes organized nuclear structures which are smaller than the overall system under consideration. I don't think anyone means anything more than that. Or at least, they probably shouldn't.

\subsection*{Alpha Clustering}
An interesting question that hasn't been immediately obvious to me is how to deal with the concept of alpha clusters. I don't see a channel on my PES corresponding to an alpha decay. Perhaps it might manifest itself as a very narrow, deep channel that doesn't show up with the resolution of my current oganesson surface. So I wanted to see what else people had come up with to predict alpha decay half-lives, and especially to see if there was anything more than just phenomenological hand waving. And it turns out the answer is yes. From what I understand, alpha decay (at least according to this thesis and the papers cited therein: http://uir.unisa.ac.za/bitstream/handle/10500/1220/dissertation.pdf) involves $p-p$ and $n-n$ pairs forming in high-lying states near the surface, with $p-n$ pairing giving the final catalyst to let a particle shed off (configuration mixing).

\subsection*{Remaining questions}
Since localizations can be used to visualize clusters, I'm going to posit a question here (which was originally suggested by Gregory Potel): Would it be possible to visualize pairs, and not just individual nucleons? If so, it might be interesting to see how pairs move around as you near scission (perhaps especially as part of a time-dependent calculation).