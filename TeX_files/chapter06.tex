\chapter{Some notes on HFODD}

Since a big chunk of my time has been spent wrestling with the DFT solver HFODD and digging around the source code and whatnot, I thought it would be a good idea to write out at least some of the useful things I've learned that should make it easier for others (or myself) down the line.

First of all, a note on the structure of the code: The actual ``solver,'' or the main function that makes up the skeleton of HFODD only goes to about line 8918 or so in the main file (depending on which version of the code you are using, of course; this figure refers to the Argonne SVN \texttt{branches/inertia} version of the code as of 23 August 2017, for whatever that's worth). A better way to identify it might be to look for the subroutine \texttt{PROANG}, and the main function ends right before that. And actually, the Hartree-Fock iterations don't actually begin until $\sim3820$, and end $\sim8365$; everything before tends to be setup and reading in data and whatnot; everything after is mostly writing data to files. The mean-field matrix diagonalization is completed by $\sim5985$; everything else inside the loop is just bonus physics calculations that take advantage of the newly-computed densities, with some error handling thrown in there as well. So don't be too intimidated by the massive size of the code base; the actual functionality - everything you need to know about the code - is only around 9000 lines (still a lot, but nothing compared to the 200,000 or so lines that make up the entirety of HFODD).