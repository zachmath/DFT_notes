\chapter{Some notes on HFODD}

Since a big chunk of my time has been spent wrestling with the DFT solver HFODD and digging around the source code and whatnot, I thought it would be a good idea to write out at least some of the useful things I've learned that should make it easier for others (or myself) down the line.

First of all, a note on the structure of the code: The actual ``solver,'' or the main function that makes up the skeleton of HFODD only goes to about line 8918 or so in the main file (depending on which version of the code you are using, of course; this figure refers to the Argonne SVN \texttt{branches/inertia} version of the code as of 23 August 2017, for whatever that's worth). A better way to identify it might be to look for the subroutine \texttt{PROANG}, and the main function ends right before that. And actually, the Hartree-Fock iterations don't actually begin until $\sim3820$, and end $\sim8365$; everything before tends to be setup and reading in data and whatnot; everything after is mostly writing data to files.

In fact, it takes a couple hundred more lines of setup inside the loop before any calculations are done, starting with the words ``\texttt{CALCULATING THE MATRIX ELEMENTS OF THE NEUTRON MEAN FIELD}'' and a call to \texttt{INTEGH} (which just adds the individual pieces together, apparently; it doesn't really perform any matrix multiplications). Likewise, it then calculates the neutron pairing field (the matrix elements of the mean-field Skyrme Hamiltonian; multipole constraints are added in there via a call to \texttt{INTCON}). It mixes the HFB matrix, saves the fields, and then diagonalizes the HFB Hamiltonian (\texttt{HFBSI*} in the HO basis, where \texttt{*} depends on symmetries requested by the user). Then it does \texttt{QUABCS}, which calculates the new Fermi energy, followed by a subroutine (for example, \texttt{CANQUA}) which finds and stores the U and V matrices, which transform between canonical and quasiparticle basis. That's another 300 lines. And so what does it do after that? Well, it spends some time computing single-particle average properties (but only when it is ``absolutely necessary''). \textbf{Then} it computes the Skyrme-functional densities and currents (\texttt{DENSHF}), the neutron density matrix, and from there, energies, magnetic moments, the RMS radius, and other properties of the system. That's all done by about line 5000, and then it changes to protons. So it's about 1000 lines of actual HFB calculations for neutrons and about another 1000 for protons (and even less than that, because some of those 1000 lines are just using the computed densities to compute other things). I guess that's not surprising; it's just building and diagonalizing a matrix. Everything else is seeing what cool things you can do with this thing you've just built.

The mean-field matrix diagonalization is completed by $\sim5985$; everything else inside the loop is just bonus physics calculations that take advantage of the newly-computed densities, with some error handling thrown in there as well. So don't be too intimidated by the massive size of the code base; the actual functionality - everything you need to know about the code - is only around 9000 lines (still a lot, but nothing compared to the 200,000 or so lines that make up the entirety of HFODD).

Another thing to note is that there is by default a constraint on $Q_{20}$. There are, of course, plenty of preset values, but this one seems particularly obnoxious. I think there's a way around it but I don't remember off the top of my head what it was. $\rightarrow$ I think you can set the stiffness parameter to zero.

In the table of energies, the so-called ``total energy'' (the very last term in the bottom right) is (so far as I can tell) given by equation I-9 (from the orignal HFODD paper; see also eq VI-96):
\begin{align}
E_{tot} &= E_{kinetic} + E_{Skyrme} + E_{Coulomb} + E_{pair} \\
& + \left[ E_{LN} + E_{pot} + E_{Gog} + E_{Yuk} + E_{CM} + E_{rot} \right]
\end{align}

\section*{Notes on specific subroutines, variables, etc. that I once found useful to know}

Subroutine \texttt{paipri} appears to be printing values it got from the replay file. It uses the variable \texttt{EFER2X} internally, but it is called with \texttt{EFER2N}.

The record file is actually read in ~3447, after some info is printed (\texttt{INFPRI}).

\texttt{ADJBAS} adjusts to find the optimal basis, given the requested value of $Q_{20}$.

\texttt{QMULCM} updates Lagrange multipliers of multipole constraints.

\texttt{QCNTRS} calculates energy contributions due to multipole contraints.

\texttt{INTCON} Calculates multipole moment constraining fields. Called inside \texttt{INTEGH}, which is where matrix elements of the Skyrme mean field are calculated (see eqn 28 of \cite{Dobaczewski1997}). Returns \texttt{HPPCON/HPMCON} (or \texttt{HAUXPP, HAUXPM} as given in the subroutine call).

\texttt{INTEGH} Calculates mean-filed Hamiltonian matrix elements. I believe, but am not positive, that this is eqn 26 in \cite{Dobaczewski1997}. Nothing complicated is really computed in this subroutine; rather, it makes calls to a bunch of subroutines, which individually calculate the different pieces of things found in eqns 27-30, 111, etc, and then just adds the results into a single matrix.

\texttt{DENSHF} Calculates densities (see eqns 40-51 in \cite{Dobaczewski1997})